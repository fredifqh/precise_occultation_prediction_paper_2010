% DO NOT COMPILE THIS FILE DIRECTLY!
% This is included by the the driver file (FlipBeamerTemplate.tex).

%---------------------- Frame de caratula ---------------------------------------

{ %% This is a total kludge for a fancy title page background
%\setbeamertemplate{sidebar right}{\llap{\includegraphics[width=\paperwidth,height=\paperheight]{BG_upper}}}
\begin{frame}[c]%{\phantom{title page}} 
% The \phantom{title page} is a kludge to get the red bar on top
% \titlepage
\begin{center}
	% \includegraphics[width=7cm]{WarpedPenguinsReturn}

%%%%%%%%%%%%%%%%%%%%%%%%%  Title Graphic  %%%%%%%%%%%%%%%%%%%%%%%%

	%\begin{tikzpicture}%[show background grid] %% Use grid for positioning, then turn off
		%\node[inner sep=0pt,above right] (title) 
			%{ \includegraphics[width=7cm]{\titleimage} };
		% \node (title) at (1.5,1.5) {};
	%\end{tikzpicture}

%%%%%%%%%%%%%%%%%%%%%%%%%%%%%%%%%%%%%%%%%%%%%%%%%%%%%%%%%%%%%%%%%
	\quad

	% \includegraphics[width=7cm]{\titleimage} 
	
	\vspace{1em}
	\normalsize\textcolor{black}{\textbf{Precise predictions of stellar occultations by Pluto, Charon, Nix, and Hydra for 2008-2015}
	\texttt{}}\\
	\vspace{.5em}
	\quad
	\footnotesize\textcolor{black}{M. Assafin, J. I. B. Camargo, et al.} \\
	\vspace{1.5cm}
	\footnotesize\textcolor{gray}{Tópicos de Ocultações no Sistema Solar, 2017 B}\quad

	%\includegraphics[height=1.5cm]{\tanedo} \quad
	% {\fontspec{Zapfino} Flip Tanedo} \quad
	% \includegraphics[height=1cm]{FlipSansSerif} \quad
	%\includegraphics[height=1.5cm]{CUasym}\\
	% \footnotesize\textcolor{gray}{In collaboration with} Csaba Cs\'aki\textcolor{gray}{,} Yuval Grossman\textcolor{gray}{, and} Yuhsin Tsai\normalsize\\
		%\footnotesize\textcolor{gray}{In collaboration with 
		%D. Grayson, J. Todd, T. Drake, S. Brown, D. Wayne}\normalsize\\
	%\textcolor{normal text.fg!50!Comment}{\textit{Tópicos}, \today}
	% \textcolor{Comment}{ \;($\pi$ day)}\\
	% \Comment{4 February 2011}
\end{center}
\end{frame}
}


\begin{frame}
\frametitle{Súmario} 
\tableofcontents
\end{frame}
%----------------------------------------------------------------------------------------------
\section{Objetivo} % (fold)

\label{sec:intro}

% section section_name (end)
\begin{frame}[c]{}{}

	\begin{block}{Objetivo}
		Predição precisa de ocultações estelares por Pluto e seus satelites Caronte Hydra e Nix. 
	\end{block}

	%\setbeamercovered{transparent}
	%\begin{enumerate}[<+->]
		%\pause 
		%\item \only<6>{\color{crimsonred}} Predição 
		%\item Acompanhamento + Atualização
		%\item Campanha + Observação da ocultação
		%\item Redução e análise dos dados 
		\begin{figure}[tb]
		\centering
		\includegraphics[width = 4cm, height = 4cm]{luas_pluto}
		%\caption{}
		\label{fig:figure1}
\end{figure}

	%\end{enumerate}


%\begin{itemize}
%\item <+-| alert@+> Robert De Niro
%\item <+-| alert@+> Brian De Palma
%\item <+-| alert@+> Gerard Depardieu
%\item <+-| alert@+> Tux
%\end{itemize}

%\begin{itemize}
%\item<2-> appears from slide 2 on
%\item<2-4> appears from slide 2 to slide 4
%\item<4> appears on slide 4
%\item<3-> appears from slide 3 on
%\end{itemize}
	
	%\begin{block}{2012 updates}
		%The new package is streamlined for nicer code. Also uses \texttt{fontspec} for \textit{XeLaTeX} support. MYou can now abuse fonts.
	%\end{block}
	
\end{frame}

\section{Introdução}
\begin{frame}[c]{Introdução}{}

\begin{itemize}
	\item The first consistent efforts for the prediction of stellar occultations by Pluto are described in Mink \& Klemola (1985) and cover the period 1985-1990.
	\item McDonald \& Elliot (2000a,b) prediction, now covering the period 1999-2009.
\end{itemize}
\pause
\begin{alertblock}{}
	Two important common limitations were the  \textcolor{black}{\textbf{astrometric precision of about only 0''.2}} and the lack of \textcolor{black}{\textbf{stellar proper motions}} leading to uncertainties on the order of the Earth radius for the predicted shadow paths. Also, these earlier predictions were degraded by \textcolor{black}{\textbf{poorer precision of older ephemerides}}, an issue which changed with the constant feed of new Pluto positions.
\end{alertblock}	
	%\begin{exampleblock}{Example Block}
		%Example block. (Potential uses: list of pros, relaxing facts)
	%\end{exampleblock}	
\pause
	\begin{block}{}
		To overcome these and other problems, we carried out an observational program at the ESO2p2/WFI instrument during 2007
	\end{block}
\end{frame}

%-------------------------------------------------------------

%\begin{frame}[c]{}
% \begin{itemize}
% \item This is \color{crimsonred}{crimsonred}\usebeamercolor[fg]{normaltext}.
% \item This is \color{charcoal}{charcoal}\usebeamercolor[fg]{normaltext}.
% \item This is \color{paleblue}{paleblue}\usebeamercolor[fg]{normaltext}.
% \item This is \color{turtlegreen}{turtlegreen}\usebeamercolor[fg]{normaltext}.
% \end{itemize}

		%\begin{columns}[t]
		%\begin{column}[T]{6cm}
		%	\begin{itemize}
		%	\item This is \textcolor{crimsonred}{crimsonred}.
		%	\item This is \textcolor{paleale}{paleale} / \textcolor{lager}{lager}.
		%	\item This is \textcolor{turtlegreen}{turtlegreen} / \textcolor{green}{green}.
		%	\item This is \textcolor{paleblue}{paleblue}.
		%	\end{itemize}
		%\end{column}

		%\begin{column}[T]{6cm}
		%	\begin{itemize}
		%	\item This is \textcolor{gray}{gray}.
		%	\item This is \textcolor{charcoal}{charcoal}.
		%	\item This is \textcolor{jeans}{jeans}.
		%	\item This is \textcolor{regal}{regal}.
		%	% \item This is \textcolor{keynotetop}{keynotetop}.
		%	\end{itemize}
		%\end{column}
		%\end{columns}
		%\vspace{1em}

%You can use the \alert{\texttt{textcolor}} command to use these, but the goal is to do things in a way where there are no calls to explicit colors, just user-adjustable values.
%\end{frame}

%-------------------------------------------------------------

\section{Stellar occultation predictions: astrometric rationale}
\begin{frame}[c]{Stellar occultation predictions}{Astrometric Rationale}

%These are some useful pre-defined color styles.
%	
%		\begin{columns}[t]
%		\begin{column}[T]{5cm}
%			\begin{itemize}
%			\item This is \alert{alert}.
%			\item This is \Alert{Alert}.
%			\item This is \ALERT{ALERT}.
%			% \item This is \alert3{alert3}.
%			\end{itemize}
%		\end{column}

%		\begin{column}[T]{5cm}
%			\begin{itemize}
%			\item This is \comment{comment}.
%			\item This is \Comment{Comment}.
%			\item This is \COMMENT{COMMENT}.
%			% \item This is \textcolor{charcoal}{comment3}.
%			\end{itemize}
%		\end{column}
%		\end{columns}
%		\vspace{1em}
%		Some colors like \textcolor{FlipGreen}{FlipGreen} and \textcolor{FlipSand}{FlipSand} will automatically change tint when when you define light or dark backgrounds. \Comment{This makes it easier to swap between light/dark backgrounds by just modifying one option and recompiling.}
%		\vspace{1em}
%		\Comment{The `comment' styles are automatically footnote-sized.}
\footnotesize{
		\begin{columns}[t]
		\begin{column}[T]{5.5cm}
			\begin{itemize}
			\item ICRS (1998)
			\item HIPPARCOS catalog (1997)
			\item 2MASS (2003)
			\item USNO B1.0 (2003)
			\item UCAC2 (2004)
			\item GSC2.3 (2008)
			\end{itemize}
		\end{column} \pause
		\begin{column}[T]{5.5cm}
				\begin{itemize}
			\item Remarkable improvement in the prediction of stellar occultations
			\item Telescopes equipped with CCDs with a relatively small FOV
			\item Better estimates for the Pluto ephemeris
			\item Pluto was entering in front of the projected galactic plane
			\end{itemize}
		\end{column}
		\end{columns}} \pause
\vspace{0.4 cm}
	\begin{block}{}
		Since 2004, our group has been engaged on a systematic effort to derive astrometric predictions for stellar occultations by Pluto and its satellites.
	\end{block}

\end{frame}

\begin{frame}
	\pause
	\begin{block}{}
		As time goes by, mostly for magnitudes fainter than about R $=$ 14, the estimation of star coordinates for current and future events is severely degraded by:
	\end{block}
	\begin{itemize}
	 	\item Increasing errors in proper motion and mean catalog position, (uncertainties of more than 70 mas for UCAC2).
	 	\item Chances are that relatively faint V or R, but bright infrared-emitting stars might be missed.
	 	\item Another issue is the problem of zero-point reference frame errors inherent to small FOV astrometry.
 \end{itemize} 

\end{frame}

%-------------------------------------------------------------

\begin{frame}
	\pause
	\begin{block}{}
		To overcome these problems, an observational program was carried out at the ESO2p2/WFI instrument during 2007.	
	\end{block}
	\begin{itemize}
	 	\item Precise positions were obtained and accurate predictions derived for stellar occultations by Pluto and its satellites Charon, Nix and Hydra.
	 	\item An astrometric catalog of 30' width was derived encompassing the 2008-2015 sky path of Pluto (20 mas).
 \end{itemize} 
\pause
\begin{alertblock}{}
Predictions of past 2005-2008 stellar occultations by Pluto and Charon were updated by an astrometric follow-up program carried out in that period at the B\&C 0.6 m telescope of the Laboratório Nacional de Astrofísica (LNA), Brazil.	
\end{alertblock}
\end{frame}

%-------------------------------------------------------------

\section{Observation at ESO}

\begin{frame}[t]{Observation at ESO}

\begin{itemize}
	\small
	\item Telescope 2.2 m (ESO2p2)
	\item CCD mosaic detector WFI (Wide Field Imager).
\end{itemize}
		\begin{figure}[tb]
		\centering
		\includegraphics[scale = 0.25]{WFI.pdf}
		%\caption{}
		%\label{fig:figure1}
		\end{figure}
\footnotesize{
		\begin{columns}[t]
		\begin{column}[T]{5.5cm}
			\begin{itemize}
				\item Exposure time 30s
				\item Broad-band R filter
				\item S/N $\approx$ 200 objetcs with R $=$ 17
				\item limiting magnitude R $=$ 21
			\end{itemize}
		\end{column} 
		\begin{column}[T]{5.5cm}
			\begin{itemize}
				\item Seeing 0"6 - 1".5
				\item September - October (2007)
				\item Pluto's sky path from 2008 - 2015
			\end{itemize}
		\end{column}
		\end{columns}}

%% Compare when you turn this on:
	% \uncover<2->
	% {
	% 	\begin{picture}(0,0)(0,0)
	% 		\put(10,-250)
	% 		{\includegraphics[width=7cm]{TRex}}
	% 	\end{picture}
	% }

%This slide demonstrates
%\begin{itemize}
%	\item \alert{absolute placement}  of images using the \texttt{put} command in the \texttt{picture} environment.
%	\item Note the overlap. Further, note that the particular depend on where the picture is defined. \\
%	\comment{If you define the picture at the top of the slide, then it will have fixed coordinates (using the [t] alignment). The cost is that the image is then behind all the text.}
%	\item Beamer respects \texttt{png} and \texttt{pdf} transparencies.
%\end{itemize} %

%\uncover<2->
%{
%	\begin{picture}(0,0)(0,0)
%		\put(10,-100)
%		{\includegraphics[width=7cm]{TRex}}
%	\end{picture}
%}%

%	\vspace{1 em}
%	\footnotesize{\Comment{Image: \url{http://www.smbc-comics.com/index.php?db=comics\&id=2109}}
%	\vspace{1 em}
%	\comment{Some alternatives for placing images:
%	\url{http://www.texample.net/tikz/examples/transparent-png-overlay/}}}
%	\normalsize
\end{frame}

%-------------------------------------------------------------

\begin{frame}
		\begin{figure}[tb]
			\centering
			\includegraphics[scale = 0.4]{table_pluto_sky_path.png}
		\end{figure}
\end{frame}
%-------------------------------------------------------------


%-------------------------------------------------------------

\begin{frame}
		\begin{figure}[tb]
			\centering
			\includegraphics[scale = 0.5]{sky_path_pluto.png}
		\end{figure}
\end{frame}

%-------------------------------------------------------------

\section{Astrometry}
\begin{frame}[c]{Astrometry}{}
	\begin{columns}[t]
	\begin{column}[T]{5cm}
	Using the esowfi and mscred packages with (IRAF):
	\begin{enumerate}
		\item Overscan
		\item Bias
		\item Flatfield
	\end{enumerate}
	\end{column}
	\begin{column}[T]{5cm}
	Using PRAIA package:
		\begin{enumerate}
			\item Field distortion pattern
			\item Astrometry over individual CCDs
			\item Global astrometric solution
		\end{enumerate}
	\end{column}
	\end{columns}
	
\end{frame}

%-------------------------------------------------------------

\subsection*{Field distortion pattern}
\begin{frame}[c]{Field distortion pattern}

		\begin{figure}[tb]
			\centering
			\includegraphics[scale = 0.3]{field_distortion_pattern.png}
		\end{figure}

The procedure for mapping the distortions for each CCD of the WFI mosaic started by superposing the observed minus catalog (O-C) position differences of UCAC2 stars computed from the respective individual CCD astrometric solutions.

\end{frame}

%-----------------------------------------------------------------------------------------------

\subsection*{Astrometry of individual CCD frames}

\begin{frame}[c]{Astrometry of individual CCD frames}
Positions were obtained with PRAIA (Assafin 2006). This fast astrometric/photometric package automatically identifies objects on the fields.
	\begin{columns}[t]
	\begin{column}[c]{5.5cm}
		\begin{figure}[tb]
			\centering
			\includegraphics[scale = 0.5]{position_function_R}
		\end{figure}
	\end{column}
	\begin{column}[c]{5.7cm}
		\begin{figure}[tb]
			\centering
			\includegraphics[scale = 0.5]{mean_error_individual_astrometry}
		\end{figure}
	\end{column}
	\end{columns}
\end{frame}

%-----------------------------------------------------------------------------------------------

\subsection*{Mosaic global astrometric solution}

\begin{frame}{Mosaic global astrometric solution}
	
\end{frame}
%----------------------------------------------------------------------------------------------
\begin{frame}[c]{Computation of proper motion}{}
One important step in our astrometric procedure was the derivation of proper motions for stars not belonging to the UCAC2, using the 2MASS and USNO B1.0 catalogs as first epoch.
\end{frame}

%-----------------------------------------------------------------------------------------------


%-----------------------------------------------------------------------------------------------

\section{The catalog of star positions along Pluto's 2008-2015 sky path}

\begin{frame}{The catalog of star positions along Pluto's 2008-2015 sky path}
	\begin{figure}[tb]
			\includegraphics[scale = 0.5]{catalog_star}
	\end{figure}
\end{frame}

%-----------------------------------------------------------------------------------------------

\section{Pluto's ephemeris offsets}


\begin{frame}[c]{Pluto's ephemeris offsets}
	
\end{frame}

%-----------------------------------------------------------------------------------------------

\section{Search procedure for candidate stars}


\begin{frame}[c]{Search procedure for candidate stars}
	
\end{frame}

%-----------------------------------------------------------------------------------------------

\section{Predictions of stellar occultations by Pluto and its satellites}

\begin{frame}[c]{Predictions of stellar occultations by Pluto and its satellites}

\end{frame}


% \begin{frame}[t]{Fancy `Handwriting' Fonts}
% 	Only works with the \texttt{emerald} package installed.
% 	\begin{itemize}
% 		\item \ECFAugie{ECFAugie. This is a rather charming font.} 
% 		\item \ECFTallPaul{ECFTallPaul. This is a pretty cramped font.} 
% 		\item \ECFJD{ECFJD. This is closest to my handwriting.}
% 	\end{itemize}
% \end{frame}


%-----------------------------------------------------------------------------------------------

\section{Discussion}

\begin{frame}[t]{Discussion}{}

\end{frame}

